Contamos con muestras de cristales de fluoruro de litio (LiF) y bromuro de potasio (KBr) y un dispositivo experimental controlado por ordenador que contiene un tubo de rayos X cuyo ánodo es de cobre. El dispositivo dirige los rayos X hacia las muestras cristalinas. Mediante el programa COBRA registraremos las medidas del espectro de rayos X para dichos cristales en función del ángulo de incidencia de entre aproximadamente 5º y 45º, haciendo cinco etapas para la toma de datos de LiF, en las que comenzamos con una diferencia de potencial de 13kV y luego iremos aumentando de tres en tres kV hasta alcanzar los 25kV. Para el KBr llevaremos a cabo una única medida a 25kV.

Todas aquellas medidas que tengan incertidumbre asociada será explícitamente mencionado. Para aquellas magnitudes indirectas se calculará el error asociado según la fórmula habitual

\begin{equation}
	\Delta{A_i} = |\frac{\partial A}{\partial \alpha_i}|\Delta \alpha_i
\end{equation}
