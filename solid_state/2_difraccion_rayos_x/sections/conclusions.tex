En esta práctica se ha estudiado la estuctura cristalina del LiF y el KBr mediante la difracción de rayos X generados por un ánodo de cobre. Mediante la aplicación directa de la Ley de Bragg hemos calculado los valores de distancia interplanar y parámetro de red, obteniendo valores muy cercanos a sus correspondientes teóricos.

Gracias a la Ley de Bragg y el entendimiento de la radiación de frenado (o bremsstrahlung) hemos podido calcular experimentalmente el valor de la constante de Planck $h = (5.5 \pm 0.9)\cdot 10^{-34} J \cdot s$. Aunque dicho valor experimental difiere notablemente del valor teórico $h = 6.626 \cdot 10^{-34} J \cdot s$, siendo el error relativo entorno a un 17\%, tenemos que el rango de incertidumbre es lo suficientemente alto como para englobar el valor real. Esta diferencia radica principalmente en el error que supone la selección visual de los ángulos mínimos de radiación de frenado.