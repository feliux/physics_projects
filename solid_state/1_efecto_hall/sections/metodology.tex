Contamos con un una lámina de p-Germanium de 1mm de grosor y 2cm de largo con una sección transversal de $10^{-5}m^2$ y un dispositivo experimental controlado por ordenador. Mediante el programa MEASURE con el plugin "Cobra3 Hall-Efect" realizaremos la toma de medidas de manera automática. Podremos configurar y variar intensidades de campo magnético, intensidad de corriente y voltaje gracias a una fuente de alimentación. Además, podremos cambiar la escala para calentar la lámina y medir así la temperatura. Todas aquellas medidas que tengan incertidumbre asociada será explícitamente mencionado. Para aquellas magnitudes indirectas se calculará el error asociado según la fórmula habitual

\begin{equation}
	\Delta{A_i} = |\frac{\partial A}{\partial \alpha_i}|\Delta \alpha_i
\end{equation}
