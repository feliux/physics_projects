Durante este experimento se han estudiado los fundamentos que subyacen al funcionamiento de una célula solar. A través de la corriente de cortocircuito y la tensión de circuito abierto se ha llevado a cabo la caracterización de la célula, así como el cálculo de parámetros importantes como son la potencia máxima, el factor de forma y la eficiencia. Nuestros experimentos han corroborado los fundamentes teóricos en cuanto a dependencia de estos valores frente a la variación de distancia entre la fuente y la célula (a través de la luminosidad que llega a la célula) y la temperatura a la que ésta está sometida.

En el caso de nuestra célula en particular, hemos verificado experimentalmente que la eficiencia es mayor cuando sobre ella incide luz solar, obteniendo un 12\% de eficiencia frente al 4\% de una lámpara incandescente. Esto es así debido a que las células solares son fabricadas de tal manera que sean capaces de aprovechar una mayor parte del espectro de la radiación solar en comparación con una lámpara incandescente.