Llevaremos a cabo la caracterización de una célula fotovoltaica de $50cm^2$. Para la toma de medidas contamos con un dispositivo experimental basado en una fuente luminosa, un amperímetro, un voltímetro, un reostato, termopila, un cristal de vidrio y un generador de aire caliente.

En primer lugar mediremos la intensidad de luz que emite la fuente luminosa en función de la distancia. Para ello primero que nada establecemos el cero de la termopila, con sensibilidad $22.69 \mu V/Wm^{-2}$, y tras fijar un factor de amplificación en 100 mediremos diferentes voltajes dados por la termopila $V_{th}$ a diferentes distancias. Una vez sepamos la intensidad de luz que llega a ciertas distancias, colocaremos la célula en sendas posiciones para variar la resistencia de la carga desde $V = V_{OC}$ (o $I = 0)$) hasta $V=0$ (o $I_{SC}$) y así medir valores de intensidad y corriente. Con estos datos calcularemos los parámetros de la célula solar.

Para comprobar las desviaciones de los parámetros en función de la temperatura repetiremos el proceso situando un cristal entre la fuente luminosa y la célula en un experimento y luego aplicaremos calor gracias al generador de aire caliente en un segundo experimento.

Finalizaremos con una comparación de la eficiencia exponiendo la célula a la luz solar frente a la eficiencia calculada en el caso de la fuente luminosa.