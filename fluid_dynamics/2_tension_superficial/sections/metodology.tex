Contamos con un dispositivo experimental compuesto por una balanza de torsión, un anillo de diámetro $d = 19.5$mm que nos servirá para medir la tensión superficial, hilo de seda, termómetro y placa calefactora.

Como método experimental se ha llevado a cabo el método de Du Nouy, el cual consiste en determinar la tensión superficial midiendo la fuerza necesaria para extraer el borde afilado de un anillo sumergido en el líquido justo en el momento en que se despega el anillo. La tensión superficial vendrá dada por

\begin{equation}
	\sigma = \frac{F}{2 \cdot 2\pi r} \label{eq_nouy}
\end{equation}

Siendo $2\pi r$ la longitud de circunferencia del anillo y donde el factor 2 se debe a que las dos superficies que limitan la película de líquido arrastrada por el anillo contribuyen a la fuerza.

Disponemos entonces de una cubeta rellena con agua destilada posada sobre la placa calefactora y en la que hemos sumergido nuestro termómetro. Suspenderemos el anillo sobre la superficie del agua por medio de un hilo de seda sujeto al brazo de la balanza de torsión y evacuaremos sucesivas cantidades de agua utilizando una pipeta. 

Debemos regular la balanza en dos momentos durante este proceso. En primer lugar, equilibraremos la balanza en su cero cuando sobre esta cuelgue el anillo y, en segundo lugar, justo después de sumergir el anillo en el agua y tras cada evacuación. Esto es debido a que la balanza se desequilibra como consecuencia de las fuerzas de tensión que inmediatamente aparecen al poner en contacto el anillo con el líquido. La lectura de la fuerza en el dial de la balanza corresponderá a la fuerza neta adicional que experiemnta el anillo, es decir, la fuerza de tensión superficial menos el empuje de Arquímedes debida a la fracción de anillo sumergido.

Cuando la extracción de agua no requiera equilibrar la balanza registraremos el valor de la fuerza que marca el dial. Repetiremos este proceso para varias temperaturas (decrecientes) gracias a la placa calefactora.

Todas aquellas medidas que tengan incertidumbre asociada será explícitamente mencionado. Para aquellas magnitudes indirectas se calculará el error asociado según la fórmula habitual

\begin{equation}
	\Delta{A} = |\frac{\partial A}{\partial \alpha_i}|\Delta \alpha_i
\end{equation}